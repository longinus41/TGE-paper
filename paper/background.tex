% !TEX root = main.tex

\section{Background}
\subsection{Blockchain and Ethereum}
Satoshi Nakamoto published Bitcoin whitepaper~\cite{Nakamoto2008} in October of 2008. As the earliest application of blockchain, Bitcoin is the most striking example of the concept of a decentralized cryptocurrency system. The production of Bitcoin depends on massive computations solving cryptographic puzzle instead of any organization, which guarantees the consistency in the distributed ledger system.

With the development of blockchain technology, more successors have merged and tried to extend the functions related to different applications. The most significant one is Ethereum~\cite{buterin2013ethereum}, providing Turing-complete smart contracts, which opens new possibilities of applications. People can develop distributed applications (DApps) with complex functions based on the Ethereum smart contract. These DApps provide the solutions for various fields other than basic transactions, such as decentralized exchange (DEX), initial coin offering (ICO), lending and so on.

However, as one of the most salient features, the anonymity of blockchain leads to that it is hard to find the real identities behind addresses as well as it protects the users' privacy. It makes trading analysis difficult and offers a breeding ground for frauds. According to Cointelegraph, the Ethereum network has experienced considerable phishing, Ponzi schemes and other scams events, accounting for about 10\% of ICOs~\cite{cerchiello2018icos}.

\subsection{Graph Embedding}
Graph analysis has been attracting increasing attention in the recent years which enables researchers to understand the various network system in a systematic manner. In the survey of graph embedding~\cite{cai2018comprehensive}, graph analytic tasks can be broadly abstracted into a lot of categories, such as node classification~\cite{bhagat2011node}, link prediction~\cite{liben2007link}, clustering~\cite{ding2001min} and visualization~\cite{maaten2008visualizing}. For example, node classification aims at determining the label of nodes based on other labeled nodes and the topology of the network.

Graph embedding provides an effective way to solve the graph analytics problem which convert the graph into a low dimensional space in which the graph information is preserved~\cite{hamilton2017representation}. In the past decade, there has been a lot of research in the field of graph embedding, and the most significant methods are factorization based methods\cite{ahmed2013distributed,belkin2002laplacian,roweis2000nonlinear}, random walk based methods\cite{perozzi2014deepwalk,grover2016node2vec} and deep learning based methods\cite{wang2016structural,kipf2016semi}. 

Embedding graphs into low dimensional spaces is not a trivial task and the challenges of graph embedding depend on the problem setting. The input of graph embedding is a graph which constructed from raw data. In \cite{goyal2018graph}, the most studied graph embedding input is heterogeneous graph which both nodes and edges belong to multiple types respectively. Typical heterogeneous graph mainly exist in the scenarios such as community-based question answering, multimedia network and knowledge graphs. According to what we have learnt, this paper is the first work to analyze the blockchain trading graph based on graph embedding techniques.·