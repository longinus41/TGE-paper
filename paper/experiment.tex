% !TEX root = main.tex

\section{Experiments}
In this section, we conduct experiments to evaluate the performance of our method via node classification on ETG.

\subsection{Data Collection and Graph Construction}
We collect all data by running Ethereum client\footnote{Parity Ethereum Client, https://www.parity.io/ethereum/} which maintains the same copy of blockchain with all historical transactions. Note that we choose the transaction logs on Ethereum from January 1, 2018 to March 31, 2018 (116,293,867 external transactions and internal transactions in total) as the input of graph construction since it is the most active period with various activities.

By parsing the transactions, 16,599,825 active accounts are obtained, including 14,450,993 EOAs and 2,148,831 SCs. Then we construct the original ETG based on these accounts and transactions.

Specially, we extend the pre-processing scheme to adapt our model. First we construct four relation graphs, which contains ETH transfer graph, contract creation graph, contract invocation graph and mining reward graph. In each graph, repeated edges between the same node pair are merged via the method introduced in section \ref{section:time}.

Last, a set of accounts with label introduced before is provided to train the model and evaluate classification accuracy. It is hard to reveal the identity of addresses since the anonymity of blockchain. We obtain these labeled examples in two ways, \emph{Etherscan}\footnote{Etherscan LabelCloud, https://etherscan.io/labelcloud} and \emph{Searchain}\footnote{Searchain, http://www.searchain.io/}. In the label set, the number of samples of each category is 100.

\subsection{Experimental Set-Up and Baselines}
As a baseline for our experiments, we compare against state-of-the-art classification results from \textcolor{red}{DeepWalk~\cite{perozzi2014deepwalk}, PARW~\cite{wu2012learning} and rGCN~\cite{schlichtkrull2018modeling}}. DeepWalk uses random walks on graphs to obtain node representations. Due to DeepWalk is an unsupervised method, a logistic regression model is added for classification. As a label propagation method, PARW is guaranteed to meet the cluster assumption under proper absorption rates. rGCN is a kind of deep neural network based method and can be applied to modeling relational data. 

Unless otherwise noted, all the GCN-based hidden layers have 16 units. Models are trained with Adam optimizer for 100 epochs, and dropout with $dropout\_rate=0.5$ is utilized to avoid overfitting.

\begin{table*}
\footnotesize
\centering
\caption{Identity Classification Results}
\resizebox{\textwidth}{17mm}{
\begin{tabular}{l|ccc|ccc|ccc|ccc|ccc}
\toprule
 & \multicolumn{3}{c|}{DeepWalk} & \multicolumn{3}{c|}{PARW} & \multicolumn{3}{c|}{rGCN} & \multicolumn{3}{c|}{rGCN+asymmetric proximity} & \multicolumn{3}{c}{Ours} \\
\midrule
& \textbf{Precision} & \textbf{Recall} & $\mathbf{F_1}$ & \textbf{Precision} & \textbf{Recall} & $\mathbf{F_1}$ & \textbf{Precision} & \textbf{Recall} & $\mathbf{F_1}$ & \textbf{Precision} & \textbf{Recall} & $\mathbf{F_1}$ & \textbf{Precision} & \textbf{Recall} & $\mathbf{F_1}$ \\
\midrule
 phish and hack & 0.609 & 0.394 & 0.479 &0.565 & 0.333 & 0.419 & 0.913& 0.212& 0.344& 0.720& 0.727& 0.724& 0.714& 0.758& 0.735\\
 token contract & 0.857& 0.735 & 0.791 &0.354& 0.718& 0.475& 0.908& 0.602& 0.724& 0.958& 0.939& 0.949& 0.958& 0.939& 0.949\\
 exchange deposit & 0.586 & 0.531 & 0.557 &0.692& 0.281& 0.400& 0.688& 0.440& 0.537& 0.615& 0.640& 0.628& 0.556& 0.600& 0.579\\
 exchange root & 0.647 & 0.759 & 0.698 &0.667& 0.759& 0.710& 0.923& 0.686& 0.787& 0.862& 0.714& 0.781& 0.828& 0.686& 0.750\\
 pool & 0.692 & 0.750 & 0.720 &0.789& 0.625& 0.697& 1.000& 0.727& 0.842& 0.842& 0.727& 0.781& 1.000& 0.727& 0.842\\
 miner & 0.400 & 0.694 & 0.508 &0.667& 0.872& 0.756& 0.867& 0.951& 0.907& 0.826& 0.927& 0.874& 0.841& 0.902& 0.871\\
 primary market & 0.405 & 0.548 & 0.466 & 0.727& 0.516& 0.604& 0.739& 0.548& 0.630& 0.680& 0.548& 0.607& 0.750& 0.484& 0.588\\
 ICO wallet & 0.364 & 0.353 & 0.358 & 0.630& 0.500& 0.558& 0.546& 0.158& 0.245& 0.769& 0.526& 0.625& 0.742& 0.605& 0.668\\
 \midrule
 average & 0.614 & 0.583 & \bf{0.585} & 0.623& 0.577& \bf{0.570}& 0.848& 0.496& \bf{0.593}& 0.806& 0.761& \bf{0.779}& 0.811& 0.764& \bf{0.782}\\
\bottomrule
\end{tabular}
}
\label{table:overall_results}
\end{table*}

All embedding and classification programs were run on the server, which includes Intel Xeon E5 CPU with 55 processors and 128GB of memory, and the GPU used for deep learning is Nvidia 1080.

\subsection{Classification Results}
In the first experiment, we test XXX on classification with the label set. Note that one account is supposed to have a single identity instead of multiple identities. Reason for this is that, people who take on many identities, tend to hold multiple accounts usually and each account is related to one identity. Besides, as as the time span of data we collected is merely three months, cases of identity conversion is negligible.

We use three indicators to evaluate each model, including precision, recall and $F_1$ score. Precision is the fraction of relevant instances among the retrieved instances, while recall is the fraction of relevant instances that have been retrieved over the total amount of relevant instances. And $F_1$ score is a measure that combines precision and recall, which is computed as
\begin{equation}
F_1=(\frac{{precision}^{-1}+{recall}^{-1}}{2})^{-1}=2\cdot\frac{precision \cdot recall}{precision + recall}
\end{equation}

In statistical analysis of label classification, the $F_1$ score is an important indicator since it is the harmonic average of the precision and recall.

Results are summarized in TABLE~\ref{table:overall_results}. Our methods (with asymmetric proximity only and with asymmetric proximity and time-density both) outperform others in average $F_1$ score. Comparing with DeepWalk and PARW, our methods achieve higher $F_1$ scores in all labels, for these methods only utilize the local structure of nodes, while our methods involve global structural information and statistic information. 

Note that rGCN achieves the highest average precision but a low average recall. Especially, rGCN has the lowest recall of category of phish and hack, which means that many normal accounts are inferred as threatening in rGCN model. Our methods improve the overall performance by adding richer information, the time of transactions, into the model.

\subsection{Asymmetric Proximity and Time-Density}
Preserving asymmetric proximity by adjusting the adjacency matrix, our method greatly improves the robustness between different labels. We observe that in vanilla rGCN, performances of different labels vary a lot, but adding asymmetric property solves the problem. % This is because in vanilla rGCN, 

Based on this, the result is better when we take time-density into consideration. Different types of accounts have diverse active time distribution, so time-density helps to distinguish between them. Also, intensive transactions reflect more typical characteristics, therefore deserve higher weights.

%%% Local Variables:
%%% mode: latex
%%% TeX-master: "analysis"
%%% End:
