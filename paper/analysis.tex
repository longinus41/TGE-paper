% !TEX root = main.tex

\section{Preliminary}
Ethereum is the largest blockchain platform that supports various transactions including cryptocurrency transferring and smart contracts both. 

This section investigates accounts and transactions which are the basic unit of Ethereum \textcolor{gray}{and they can represent nodes and edges respectively to construct a graph constructed later.} 

\subsection{Accounts}
Accounts play a central role in Ethereum. There are two types of accounts in Ethereum, \emph{Externally Owned Accounts} (EOAs) and \emph{Contract Accounts} (CAs). 

The major difference between is that EOAs are controlled by people who hold the public-private key pairs whereas CAs are ruled by executable codes inside. For a CA, code execution is triggered by transactions or messages (calls) received from other contracts. Other than that, they do not look a whole lot different in Ethereum system. Both of them have unique addresses and \emph{Ether} (the currency used within Ethereum, short as ETH) balance.

The state of all accounts is the state of the Ethereum network which is updated with every block and which the network really needs to reach a consensus about. Accounts are essential for users to interact with the Ethereum blockchain via transactions.
 
 \subsection{Transactions}
 The term \emph{transaction} is used in Ethereum to refer to the signed data package that stores a message to be sent from an account to another account on the blockchain.
 
\textcolor{gray}{A transaction is the \emph{external one} if it is sent from an EOA while the \emph{internal one} results from executing a smart contract due to an external transaction~\cite{chen2018infocom}.}

%Note that an external transaction may lead to many internal transactions. 
%Transaction represents an action between two accounts, which is basic unit in each Ethereum block. 
%Transaction can be categorized as external one and internal one based on the sponsor of the transaction. 
Transactions can be categorized based on the method type, including \texttt{CALL}, \texttt{CRATE}, \texttt{REWARD} and \texttt{SUICIDE}. The most common of these is \texttt{CALL} transaction which comes with ETH transferring or contract invoking. \texttt{CREATE} is used to deploy smart contracts while the contract can be destroyed at the end of it's cycle via \texttt{SUICIDE} method. \texttt{REWARD} transaction appears on the head of each block, which depicts the reward that block miner obtained from special system account.

 %The sender of \texttt{REWARD} transaction is a special address \texttt{0x00...00}. In the  transaction,  \texttt{CALL}, \texttt{CREATE} and \texttt{SUICIDE} transactions can be either external transactions or internal transactions since the initiator can be both EOA and SC.
 
%For ETH transfer, the transaction value is a non-zero number. However, the process of ERC-20 token transfer is more complicated. The sponsor A (also called initiator) make a \texttt{CALL} transaction to the ERC-20 smart contract to tell that he want to transfer ERC-20 token to somebody B. Then the smart contract will check the request and complete the deal if A has enough ERC-20 token. Note that the target address of the \texttt{CALL} transaction is the ERC-20 SC instead of B address and the transfer value is $0$ ETH since the actual transfer happens in the smart contract where the ERC-20 token is transferred from A account to B account in the smart contract inner database. 

Based on the above mentioned transactions, various activities can be realized on Ethereum, such as money transfer, contract creation and contract invocation~\cite{chen2018infocom}. 

%\textcolor{gray}{The on-chain assets include ETH coin and ERC-20 token.} 


 
\subsection{Identities}
The Ethereum hosts many different types of users. Most are ordinary people, but the bulk of holdings are distributed among specific categories. 

TokenAnalyst listed several ones, including  mining pools, exchanges, token wallets, investors and whales\footnote{``Classifying Ethereum users using blockchain data'', https://www.tokenanalyst.io/}. But, frankly, the complete categorization does not exist since people intentionally make their identities harder to track and implementation of smart contract is being extended and deepened.

In the paper, we try to classify these accounts into the following prominent ones.

%However, anonymity of blockchain makes the identification more difficult. Generally, these accounts can be classified into different categories according to user roles on the blockchain.

\noindent 1) \emph{Miners \& Mining Pools}

The Ethereum blockchain is in many ways similar to the Bitcoin blockchain, in which a block is only valid if it contains \emph{Proof-of-Work} (PoW) of a given difficulty\footnote{Note that in the Ethereum Serenity milestone, this is likely going to be replaced by a Proof-of-Stake mechanism.}. The \emph{miners} are the individuals or groups who validate transaction information by solving the cryptographic puzzles. Whoever is the first to find a valid hash of block will get the reward in the form of ETH which is paid by users sending transactions.

In the early stage, most people take part in the mining process independently. With the more participants into this high profit industry, the mining competition gradually becomes fiercer. And an efficient solution is working together to solve the PoW problems. Miners with mining machines can register on a special institution named \emph{mining pool} where aggregates all the registrants' computing power to solve mining problem and distributes the reward to the registrants according to their proportion of contributed computing power. As of September 2018, top $3$ mining pools takes more than 65\% of hash rate in Ethereum\footnote{Investoon, https://investoon.com/charts/mining/eth.}.
%The centralized institution is \textbf{mining pool} and these registrants are so called \textbf{cooperative miners}.


% And, more remarkable, the percentage of standalone miners declined and dropped to almost zero in the end of 2017. 
\noindent 2) \emph{Exchanges}

The exchanges are the platforms for trading between ETH, fiat money (e.g., USD) and even other digital currencies, which play an important role in Ethereum ecosystem. The exchanges can be categorized into centralized exchanges and decentralized exchanges (also known as DEXs) according to their architectures.

Most of the world’s cryptocurrency trading is done through centralized exchanges such as Binance, Huobi, OKEX etc\footnote{``Top 100 Cryptocurrency Exchanges by Trade Volume'', https://coinmarketcap.com/rankings/exchanges}. The centralized exchange allocates a deposit address to each user who wants to make transaction in the exchange. These addresses are called \emph{exchange deposits} and belong to the exchange since users do not have the private key of these addresses. In recharge process, user transfers coins to the given deposit address from her own wallet and these coins will be transferred to the \emph{exchange root} address automatically. In turn, users send requests to exchange to withdraw their coins from an address called \emph{exchange withdrawal}. And in most cases, the exchange root and exchange withdrawal mean the same address.

%The exchange charges the user a commission for both recharge and withdraw services.

The DEXs are a new technology that facilitate cryptocurrency trading on a distributed ledger. Being completely on-chain, all orders interact with each other directly through the blockchain. This makes it fully decentralized, but also expensive and slow. Besides, another difference is that user will get a new address with corresponding private key when registers to the DEX, which means the address belongs to user itself instead of exchange. 
 
 %User calls the smart contract in the decentralized exchange address to start a transaction. Intuitively, the transaction will take a long time for making match and confirming compared with in centralized way.

%Since most ERC-20 token transactions happen via smart contracts and in a decentralized mechanism, some big exchanges which support both ETH and ERC-20 token transaction (e.g., Binance and Huobi) are mixture of centralized and decentralized exchange. 


\noindent 3) \emph{ERC-20 \& ICO}

ERC-20 is a technical standard used for smart contracts on the Ethereum blockchain for implementing tokens~\cite{erc-20-wiki}. It defines a common list of rules that an Ethereum token has to implement, giving developers the ability to program how new tokens will function within the Ethereum ecosystem. Such ERC-20 token transfer happens in specific CA which is called \emph{ERC-20 token contract}. %and the transferring process will be illustrated later. 

The ERC-20 token standard became popular with crowdfunding companies working on ICO cases due to the simplicity of deployment, together with its potential for interoperability with other Ethereum token standards~\cite{erc-20}. As of July 26 2018, there were more than 103,621 ERC-20 token contracts\footnote{Etherscan Token Tracker Page, https://etherscan.io/tokens}. Among the most successful ERC-20 token sales are EOS, Filecoin, Bancor, Qash, and Nebulas, raising over 60 million each\footnote{``Token Data, data and analytics for all ICO's and tokens", https://www.tokendata.io}.

Participants in the initial ICO round are \emph{investors} who buy the ERC-20 token from ERC-20 smart contracts of the crowdfunding companies. And these addresses where ETH holding of token teams are \emph{ICO wallets}.

\noindent 4) \emph{Phishes \& Hacks}

Since virtual property transactions are now becoming increasingly commonplace and that leads to many security issues. At the same time, the frauds associated with ETH and ERC-20 tokens have also increased. We call these addresses related to frauds \emph{phishes \& hacks}.

 In Ethersacan, there are more than $2500$ addresses are labeled as Phish/Hack, which takes up the highest proportion. Most of them are disguised as ERC-20 token sales or DApps such as casino. 

\begin{table}[htbp]
\caption{Typical Account Identities}
\begin{center}
\begin{tabular}{|p{2.1cm}|c|p{3.9cm}|}
\hline
\textbf{Identity} & \textbf{Account Type}& \textbf{Description} \\
\hline
Miners & EOA & Nodes who take part in the block validation process. \\ \hline
Mining Pools & EOA & The pooling of resources by miners, who share their processing power over a network.\\ \hline
Token Contracts & CA & Contracts that allow customers to transfer ERC-20 tokens. \\ \hline
Investors & EOA & Large holders of ETH, who usually got in early on ICOs. \\ \hline
ICO Wallets & EOA\&CA & ETH holdings of token teams, typically raised from ICOs. \\ \hline
Exchange Deposits & EOA\&CA & Addresses for user to deposit ETH at exchange. \\ \hline
Exchange Roots\&Withdrawals & EOA\&CA & Addresses collect ETH from deposit addresses and withdraw ETH to users. \\ \hline
Phishes\&Hacks & EOA\&CA & Fraud address related to phishing and hacks. \\ \hline
%\multicolumn{4}{l}{$^{\mathrm{a}}$Sample of a Table footnote.}
\end{tabular}
\label{tab1}
\end{center}
\end{table}



%Phishing is the name given to the latest online scam where millions of unwary Americans are getting their identities stolen.

%We've seen increased use of sophisticated forms and letterhead to send what appears to be legitimate World Bank Group correspondence, as well as several schemes that reference the Bank.





