% !TEX root = main.tex

\section{Preliminary}
\label{sec:preliminary}
\textcolor{red}{As the largest blockchain platform that supports various transactions including cryptocurrency transferring and smart contracts~\cite{chen2018infocom}, Ethereum is designed based on the \emph{Account/Balance} model which is more intuitive compared to the \emph{UTXO} model.}

\textcolor{red}{This section investigates Ethereum from a logical perspective, including accounts and transactions. Then the identity inferring problem is raised.}

\textcolor{blue}{Note that our work is general to blockchains that support
smart contracts, not only Ethereum. However, terminologies in different
blockchains vary a lot. Thus, we use terminologies in Ethereum in the
following. And it is practical to apply our approach to other blockchains.}

\subsection{Accounts}
Accounts play a central role in Ethereum. There are two types of accounts in Ethereum: \emph{Externally Owned Accounts (EOAs)} and \emph{Contract Accounts (CAs)}. Both EOAs and CAs have unique addresses and \emph{Ether} (the cryptocurrency of Ethereum, or ETH for short) balance. Their major difference is that EOAs are controlled by people who hold the public-private key pairs, whereas CAs are ruled by executable codes of smart contracts. For a CA, code execution is triggered by transactions or messages (calls) received from other contracts.

In summary, the state of all accounts determines the state of the Ethereum network, which is updated with every block and the network really needs to reach a consensus about. Accounts are essential for users to interact with the Ethereum blockchain via transactions.

 \subsection{Transactions}
 The term transaction is used in Ethereum to refer to the signed data package that stores a message to be sent from an account to another on the blockchain. A transaction is called the \emph{external one} if it is sent from an EOA, and the \emph{internal one} is generated by executing a smart contract by an external transaction~\cite{chen2018infocom}.

%Note that an external transaction may lead to many internal transactions.
%Transaction represents an action between two accounts, which is basic unit in each Ethereum block.
%Transaction can be categorized as external one and internal one based on the sponsor of the transaction.
Transactions can be categorized into different types, including \texttt{CALL}, \texttt{CRATE}, \texttt{REWARD} and \texttt{SUICIDE}. The most common type is the \texttt{CALL} transactions used for ETH transferring or contract invoking. The \texttt{CREATE} transactions are used to deploy smart contracts while the contract can be destroyed at the end of their cycles via the \texttt{SUICIDE} method. The \texttt{REWARD} transactions appear on the head of each block, which depicts the reward that block miner obtained from special system accounts.

 %The sender of \texttt{REWARD} transaction is a special address \texttt{0x00...00}. In the  transaction,  \texttt{CALL}, \texttt{CREATE} and \texttt{SUICIDE} transactions can be either external transactions or internal transactions since the initiator can be both EOA and SC.

%For ETH transfer, the transaction value is a non-zero number. However, the process of ERC-20 token transfer is more complicated. The sponsor A (also called initiator) make a \texttt{CALL} transaction to the ERC-20 smart contract to tell that he want to transfer ERC-20 token to somebody B. Then the smart contract will check the request and complete the deal if A has enough ERC-20 token. Note that the target address of the \texttt{CALL} transaction is the ERC-20 SC instead of B address and the transfer value is $0$ ETH since the actual transfer happens in the smart contract where the ERC-20 token is transferred from A account to B account in the smart contract inner database.

Based on the above mentioned transactions, various activities can be enabled on Ethereum, such as money transfer, contract creation and contract invocation~\cite{chen2018infocom}.

%\textcolor{gray}{The on-chain assets include ETH coin and ERC-20 token.}


\subsection{Identity Inferring}
\textcolor{blue}{
The goal of identity inferring is to determine user's role with limited
information. Specifically, we try to achieve this goal only with only the data
on blockchain. And this is largely different from some works that can leverage
additional data sources, like user profiles beside their public-keys.
}

\textcolor{blue}{
Intuitively, the fact behind identify inferring is that different identify have
different transaction patterns. For example [xxx some work that leverage some
patterns].
}

\textcolor{blue}{
However, smart contracts extend blockchain to some new users. As smart
contracts are Turing-complete, there are some smart contracts that for totally
different purpose from currencies. For example, ERC20, which is for
crowdfunding. As expected, users with new identities start experiencing
blockchain. In the ERC20 example, investors and ICO wallets start using
Ethereum, which is quite different from Bitcoin.
}

\textcolor{blue}{
It is worthwhile to mention that it is very hard, if not impossible, to make a
sound identity list. Generally, these accounts can be classified into different categories according to user roles on the blockchain. TokenAnalyst listed several prominent ones, including  mining pools, exchanges, token wallets, investors and so on\footnote{TokenAnalyst, https://www.tokenanalyst.io/}.
However, Ethereum is still in a development stage. The implementation of DApps (decentralized applications) is being extended and deepened and various identities behind such DApps emerge inevitably.
}

\textcolor{blue}{Deanonymization, particularly identity inferring remains a challenge due to complex interaction on Ethereum. Here we try valiantly to present a crude but effective taxonomy of Ethereum accounts, which are illustrated in TABLE~\ref{table:identity}. A detailed description is given in Section~\ref{sec:experiments}.  }

\textcolor{red}{The Ethereum hosts many different types of users which hold the accounts as their avatars. Most are ordinary people, but the bulk of holdings are distributed among specific categories. Generally, these accounts can be classified into different categories according to user roles on the blockchain. TokenAnalyst listed several prominent ones, including  mining pools, exchanges, token wallets, investors and so on\footnote{TokenAnalyst, https://www.tokenanalyst.io/}.}


\textcolor{red}{To be frank, the perfect categorization does not exist because Ethereum is still in a development stage. The implementation of DApps (decentralized applications) is being extended and deepened and various identities behind such DApps emerge inevitably.}

\textcolor{red}{Nonetheless, we advocate that identity inferring may also be required for transaction analysis on Ethereum. It is important to reveal the identities behind accounts without violating user privacy rights. By tracking and inferring these accounts, we can have better understanding what is occurring on the blockchain. For instance, cryptocurrency team can make airdrop to the top valuable accounts and the fraud or cheating actions can be detected efficiently.}

\textcolor{red}{Deanonymization, particularly identity inferring remains a challenge due to complex interaction on Ethereum. Here we try valiantly to present a crude but effective taxonomy of Ethereum accounts, which are illustrated in TABLE~\ref{table:identity}. A detailed description is given in Section~\ref{sec:experiments}.  }

%However, anonymity of blockchain makes the identification more difficult. Generally, these accounts can be classified into different categories according to user roles on the blockchain.

