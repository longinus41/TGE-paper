% !TEX root = main.tex
nnn
\section{Introduction}
%\textcolor{red}{TBA.}

As the second major currency on the market with more than 10 billion dollars\footnote{At the time of this writing, the capitalization of Ethereum is about 11.8 billion dollars, once as high as more than 40 billion dolloars.} of capitalization, Ethereum is developed as a public, open-sourced blockchain-based platform~\cite{buterin2013ethereum}. Unlike Bitcoin, which was designed to be a currency, Ethereum aims to facilitate software processing using a token system called \emph{Ether} (ETH). Ethereum has the ability for people to build Turing-complete smart contract, which is a collection of code and data state that resides at a specific address on the Ethereum blockchain. Based on that, Ethereum can be applied to various scenarios beyond monetary exchange, such as voting, crowdfunding, lending, property rights protection and so on.

From a logical perspective, the core elements in Ethereum include accounts and transactions. There are two types of accounts, namely \emph{Externally Owned Accounts} (EOAs) and \emph{Contract Accounts} (CAs). Both have an ETH balance and the major difference between is that EOAs are controlled by people who hold the public-private key pairs whereas CAs are ruled by executable codes inside. The transaction is used in Ethereum to refer to the signed data package that stores a message to be sent from an account to another account on the blockchain. In Ethereum, various activities, such as money transfer, smart contract creation and invocation can be leveraged via transactions~\cite{chen2018infocom}.

Generally, accounts and transactions depict the anonymous users and their relationships. The Ethereum blockchain hosts many different types of users and it is necessary to concern about the identities behind their on-chain transaction activities. By tracking and inferring these accounts, we can have better understanding what is occurring on the blockchain. Many studies focus on characterizing blockchain system through graph analysis, such as deanonymization, and money laundering detection on Bitcoin~\cite{zhao2015graph,maesa2016analysis}. Chen et al.~\cite{chen2018infocom} conduct an investigation on Ethereum by leveraging graph analysis.

On the one hand, most existing methods, however, are in the feature engineering manner, which relies on experts to exhaustive enumerate all related features. This way  depends highly on the expertise. You can hardly expect the specialist familiar with all type of graphs, let alone ordinary users. Consequently, it's mechanical, rigid and time-consuming. On the other hand, previous methods exploit merely  the node-level features and overlook the more important part, the relationship between nodes, namely the graph structural information. Due to  the anonymity feature of Ether um, little  information is revealed in the node data, which result in the poor performance of these methods in blockchain data.    Moreover, the  inherent characteristic of blockchain data, that is the dynamic evolving, or more specifically, continuous growth in size, pose more challenges to the task. The work \cite{chen2018infocom} analyze the Etherum   from  July, 2015 to June, 2017, which have only () transactions. Sadly, as the date of today, the transactions of the time period from Jan,2018 to March, 2018 have exploded into , which can be hardly handled using traditional methods.

% 2. without considering the graph structural information.

% 3. Ethereum data explosive growth, resulting in an increase of the difficulty in graph analysis. %The Ethereum transaction data we collect from January 1, 2018 to March 31, 2018 is XX times than the data in~\cite{chen2018infocom} which is from
%July 30th, 2015 to June 10th, 2017.


 In this paper, we construct a Ethereum transaction graph (ETG for short) and analyze graph based on graph embedding techniques.

 Graph embedding provides an effective way to solve the graph analytics problem which convert the graph into a low dimensional space in which the graph information is preserve~\cite{cai2018comprehensive}. Based on the low dimensional vectors, some analytic tasks can be done effectively. Here we aim at determining the identity of accounts on other labeled accounts and the topology of the transaction graph. According to what we have learnt, this paper is the first work to analyze the transaction graph of cryptocurrency based on graph embedding techniques.

At the same time, we find that using such embedding techniques directly achieves poor effect on transaction graph. This is due to Ethereum transaction graph has many properties different from traditional networks, such as social media networks and citation graph. In Ethereum transaction graph, edges stand for different activities such as money transfer, contract creation and invocation, which can not be measured in a uniform weighted model. Besides, compared with other graphs, closeness features such as second-order proximity and asymmetric proximity in ETG are characteristic, which should be preserved. We also find that the relationship between a pair of nodes differs from another, even they have the same transaction frequency which could be expressed by time-density information.

Generally, our approach consists of three phases, \emph{graph construction}, \emph{graph embedding} and \emph{node classification}. In the construction phase, we construct the Ethereum transaction graph and divide it into multi-relation graphs to capture the different graph neighboring of transactions. In the embedding phase, we propose a XXX graph convolutional network and put the features of nodes and edges into representation. Last, a semi-supervised method is used for accounts classification.

In summary, this paper make the following major contributions.

(1) We propose a taxonomy of Ethereum accounts. Even the perfect categorization does not exist due to the blockchain anonymity and diversity in Ethereum. However, it is necessary to concern about the identities behind their on-chain transaction activities.
%we can still glean insight from examining the addresses with the most substantial balances by reviewing their on-chain transaction activity.

(2) To the best of our knowledge, we are the first to analyze the transaction graph of cryptocurrency based on graph embedding techniques. We propose a three-pronged approach to handle account prediction task and reveal the differences in analysis between Ethereum transaction graph and other traditional graphs.

(3) To address these challenges, we introduce several techniques in data preprocessing and embedding. The evaluation through real data set shows the effectiveness of new approach.

The rest of the paper is organized as follows. Section~\ref{sec:background} introduces the background knowledge of Ethereum related studies in graph analysis. Section~\ref{sec:preliminary} investigates accounts and transactions, then a taxonomy of Ethereum accounts is proposed. Section~\ref{sec:graph_analysis} gives an overview of our analysis procedure and illustrates the challenges. \textcolor{red}{TBA: In Section~\ref{sec:model}}

%In Section~\ref{sec:model}, the detailed architecture
