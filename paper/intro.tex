% !TEX root = main.tex
\section{Introduction}
%\textcolor{red}{TBA.}
%\textcolor{red}{
%Today cryptocurrencies have become a global phenomenon with the market capitalization of more than xxx billion dollars. 
%Blockchain technologies are gaining massive momentum in the last few years, largely due to the success of cryptocurrencies like Bitcoin and other altcoins. Blockchain is immutable ledger for recording transactions in append-only way, maintained within a distributed network. Pseudonymous is the born feature of blockchain, i.e., each account is associated with a public address, yet its actual identity is unknown. 
Blockchain technologies are gaining massive momentum in recent years, mainly due to the success of cryptocurrencies like Bitcoin. A key feature of cryptocurrencies is the anonymity, i.e., users are identified by public-keys only, so that transactions can be done pseudonymously, without exposing real identities~\cite{reid2013analysis}.
%}

%\textcolor{red}{
%Identity inferring, or deanonymization, plays an important role in blockchains. 
On the other hand, deanonymization also attracts wide attentions. By inferring and tracking these anonymous accounts, people like traders, law-makers, financial security officers, etc, can have a better understanding about what is occurring on the blockchain. Due to its significance, deanonymization has been extensively studied, ranging from usual trading detection~\cite{maesa2016analysis} to exchange pattern mining~\cite{ranshous2017exchange}.
 
%There are many people, including traders, law-makers, financial security officers, etc, needs understand the behaviours on blockchains. 
%}.

%\textcolor{red}{
New challenges of deanonymization arise as the popularity of Ethereum, which is a blockchain-based software platform that enables smart contracts~\cite{buterin2013ethereum}.
With \textcolor{red}{Turing-complete} smart contracts, users could perform arbitrary behaviours on blockchain, instead of simple assets transfer. 
Identify inference becomes more challenging in Ethereum because of the diversity of users and complexity of their activities. Furthermore, decentralized applications (DApps) are constantly expanding and adding new user roles to blockchain ecosystem, which are unforeseeable without expert knowledge and external information.
For example, more than $2500$ accounts are labeled as ``Phish/Hack'' in Etherscan\footnote{https://etherscan.io/accounts?l=Phish/Hack.}, which are fraud addresses related to phishing or hacking in Ethereum. Many of them are disguised as ICO (Initial Coin Offering) \textcolor{red}{wallets} or DApps like casino~\cite{cerchiello2018icos}, and it is hard to detect them without source code analysis~\cite{jiang2018contractfuzzer}. 

It is not trivial to address these challenges. Existing methods~\cite{maesa2016analysis,ranshous2017exchange,zhao2015graph} have studied discerning transaction flows in currency-trade blockchains only, like Bitcoin, which is far from achieving deanonymization in identity-rich blockchains like Ethereum. A proximate work~\cite{chen2018infocom} infers the account identity by analyzing source codes (e.g., comments or keywords) of smart contracts. However, it relies on experts to exhaustively enumerate all related features, which is tedious, time-consuming and with low accuracy.

%\textcolor{red}{A practical approach is abstracting cryptocurrency transactions as a weighted directed graph where nodes depict the accounts and edges stand for various activities between account pairs. However, most existing graph analytics methods suffer the high computation and space cost on such huge amount of blockchain data.~\cite{cai2018comprehensive}.  }

In this paper, we propose I$^2$GL, an \textit{Identity Inference approach based on Graph deep Learning} to address the challenges of deanonymization in Ethereum \textcolor{red}{and other DApp platform blockchains}. The basic idea of I$^2$GL is to construct a transaction graph, where nodes denote user accounts and edges describe transactions among them. The identity of some nodes in the transaction graph are unknown. The goal of I$^2$GL is to infer the identify of these unknown nodes \textcolor{red}{by extracting graph features.} 

\textcolor{red}{Graph analytic methods have achieved great successes on social media networks~\cite{geng2015learning} and knowledge graphs~\cite{bollacker2008freebase}. However, the transaction graph is thousands times larger than traditional networks\footnote{The Ethereum transaction graph used in our experiments contains 16,599,825 nodes and 116,293,867 edges while the \emph{Citeseer}, a typical dataset of citation network, contains only 3,327 nodes and 4,732 edges.}, thus most existing methods suffer the high computation and space cost on such transaction graph. Graph embedding provides an effective way to solve the graph analytics problem which convert the graph into a low dimensional space in which the graph information is preserved~\cite{hamilton2017representation}.
Specifically, I$^2$GL adopts Graph Convolutional Network (GCN), a powerful deep learning based embedding method designed to work directly on graphs by leveraging the graph structure as convolutional layers.}

%Unlike existing work, we transfer the identity inferring task into node classification, which is a typical graph analytics problem~\cite{cai2018comprehensive}. Graph deep learning is effective for solving such problems by converting the graph into a low dimensional vectors~\cite{hamilton2017representation,battaglia2018relational}. \textcolor{red}{Based on such vectors, node classification can then be computed efficiently with high accuracy.} 
%As a general analytic approach for cryptocurrencies, AIGL consists of three phases, \emph{graph construction}, \emph{graph embedding} and \emph{node classification}. In the construction phase, we construct a transaction graph and divide it into multi-relation graphs to capture the different activities of transactions. In the embedding phase, we propose a graph convolutional network (GCN) and put the features of nodes and edges into representation. Last, a semi-supervised method is used for nodes classification.

 By carefully examining the \textcolor{red}{transaction graph}, we find that it has many unique features different from these traditional networks, which can be exploited to further improve the identity inference performance. In Ethereum, there exists multiple types of activities, e.g., money transfer, contract creation and invocation, among nodes. Therefore, instead of using a single edge with weight to describe node relationship in traditional graph, we create multiple edges between a pair of nodes in \textcolor{red}{transaction graph}, each of which represent a type of activities. Besides, we observe that even though two types of accounts conduct the same amount of transactions during a period, the transaction density over time is different. I$^2$GL enhances GCN by integrating this time-density feature into the graph learning process. Other features such as second-order proximity and asymmetric proximity in cryptocurrency transaction graph are also considered. The main contributions of this paper is summarized as follows.

(1) To the best of our knowledge, we are the first to apply graph deep learning based on GCN for identity inference on Ethereum. 

(2) Motivated by special features of \textcolor{red}{transaction graph}, we propose several novel designs to enhance the performance of traditional GCN.

(3) Experiments on Ethereum transaction records are conducted to evaluate the performance of I$^2$GL. The results show that our proposal significantly outperform existing approaches.

%\textcolor{red}{(2)  The prominent identities are illustrated and provided for model training.}

%. Even the perfect categorization does not exist due to the blockchain anonymity and diversity in Ethereum. However, it is necessary to concern about the identities behind their on-chain transaction activities.
%we can still glean insight from examining the addresses with the most substantial balances by reviewing their on-chain transaction activity.

%(3) To address these challenges, we introduce several techniques in data preprocessing and embedding. 
The rest of the paper is organized as follows. Section~\ref{sec:preliminary} introduces background knowledge of cryptocurrency and raises identity inferring problem. Section~\ref{sec:approach} gives an overview of our analysis procedure first, then detailed designs are presented. Section~\ref{sec:experiments} is experimental results.  Related work is reviewed in Section~\ref{sec:relatedworks}, followed by Section~\ref{sec:conclusion} that concludes this paper.



